\documentclass{beamer}

\mode<presentation> {
  \definecolor{frameheadforeground}{RGB}{169,33,62}
  \definecolor{frameheadbackground}{RGB}{255,255,255}

  \usetheme{Warsaw}
  %\setbeamercovered{transparent}
}

\setbeamercolor{structure}{fg=frameheadforeground,bg=frameheadbackground}

\usepackage[utf8]{inputenc}
\usepackage[MeX]{polski}

\begin{document}

\begin{frame}
  
\title[Tytuł]{Analiza wielowymiarowa}
\subtitle{Wprowadzenie}
\author{Dorota Celińska-Kopczyńska, Paweł Strawiński}
\institute{Uniwersytet Warszawski}
\date{Zajęcia 1 \\ 8 października 2019}

\titlepage
\end{frame}
\begin{frame}[allowframebreaks]
\frametitle{Plan zajęć}
  \tableofcontents
\end{frame}

\section{Sprawy organizacyjne}

\begin{frame}{Prowadzący}
  \begin{itemize}
  \item dr Dorota Celińska-Kopczyńska
    \begin{itemize}
     \item mail: dot@mimuw.edu.pl
     \item strona: \url{mimuw.edu.pl/~dot}
     \item dyżur: piątek 12:00, MIM, p. 1700, po umówieniu e-mailem
    \end{itemize}
  \item dr hab. Paweł Strawiński
    \begin{itemize}
    \item mail: pstrawinski@wne.uw.edu.pl
    \item strona: \url{coin.wne.uw.edu.pl/pstrawinski}
    \item dyżur: wtorek 17:15, WNE, p. 304, po umówieniu e-mailem
    \end{itemize}
  \end{itemize}
\end{frame}

\begin{frame}{Kontakt mailowy}
  \begin{itemize}
  \item W celu utrzymania porządku podczas kontaktu z Państwem prosimy o stosowanie formatu tytułu maili:\\
    \begin{center}
      AW\_Nazwisko\_skrót\_sprawy
    \end{center}
  \item Niedopuszczalne jest rozpoczynanie zupełnie nowego tematu (np. wysłania propozycji tematu pracy zaliczeniowej) w wątku dotyczącym innej sprawy
  \item \textbf{Kontakt wyłącznie z wykorzystaniem adresów poczty uniwersyteckiej}
  \end{itemize}
\end{frame}

\begin{frame}{Forma zajęć}
  \begin{itemize}
  \item Zajęcia w formie konwersatorium, 14 spotkań w semestrze zimowym
  \item Wymagana rejestracja w USOS
  \item Przyjście na inną grupę niż jest się zapisanym możliwe po wcześniejszym powiadomieniu mailowo prowadzących i~uzyskaniu ich zgody (w zależności od liczby wolnych miejsc)
  \item Nie dopuszczamy do sytuacji, gdy liczba uczestników jest większa niż komputerów :)
  \end{itemize}

\end{frame}

\section{Zasady zaliczenia}

\begin{frame}{Elementy zaliczenia}
  \begin{enumerate}
  \item Obecność: obowiązkowa, dopuszczalne co najwyżej 3 nieobecności, powyżej NK
  \item Raport z przeprowadzonego badania: 60\%
  \item 30 minutowa prezentacja wygłoszona w trakcie ostatnich 3~zajęć: 15\%
  \item Prace domowe: 25\%
  \end{enumerate}
  %\begin{alert}{Uwaga!}
  %  Osoby, które nie oddadzą prac w wymaganych terminach, nie otrzymają zaliczenia
  %\end{alert}
\end{frame}

\begin{frame}{Prace domowe}
  \begin{itemize}
  \item 2 prace domowe (jedna techniczna, druga interpretacyjna) z~ok.~3~tygodniowym terminem oddania
  \item Rozwiązania indywidualne -- Państwo decydują, czy chcą rozwiązać
  \item Brak punktów z prac domowych nie uniemożliwia zaliczenia przedmiotu, ale nie można też uzyskać oceny końcowej wyższej niż 4 (db)
  \item Prace domowe stanowią element Państwa aktywności -- nie poprawiamy ich w drugim terminie
  \end{itemize}
\end{frame}

\begin{frame}{Raporty i prezentacje}
  \begin{itemize}
  \item Wykonanie badania w grupach. Preferujemy grupy 3 osobowe. Prosimy unikać grup składających się wyłącznie z osób z II roku studiów II stopnia
  \item W każdej grupie powinno powstać co najmniej 7 zespołów, ale co najwyżej~9
  \item Można dobrać się w zespół z osobami z grupy z innej godziny
  \item W razie problemów w zespole, prosimy nas informować, postaramy się znaleźć rozwiązanie
  \end{itemize}
\end{frame}

\begin{frame}{Raporty i prezentacje -- tematyka}
  \begin{itemize}
  \item Tematy do wyboru z poniżej listy, zgłaszane mailowo \underline{do obojga prowadzących}:
    \begin{enumerate}
    \item analiza korelacji i analiza ANOVA
    \item analiza korespondencji i analiza korelacji
    \item analiza kanoniczna
    \item analiza dyskryminacji
    \item metody segmentacji
    \item analiza czynnikowa, inne techniki redukcji wymiaru ($\star$)
    \item analiza conjoint
    \end{enumerate}
  \end{itemize}
   \begin{alert}{UWAGA!}
     W każdej grupie wszystkie tematy muszą zostać wyczerpane, o~przydzieleniu tematu decyduje kolejność zgłoszeń.
   \end{alert}
\end{frame}

\begin{frame}{Zgłoszenia tematów}
  \begin{itemize}
    \item Zgłoszenie powinno zawierać: wybraną technikę badawczą, hipotezy/cel,
      opis źródła/bazy danych, propozycję literatury
    \item Zgłoszenie może być:
      \begin{enumerate}
      \item \textbf{zakceptowane} (odpowiedź ``akceptuję temat'' od prowadzącego)
      \item \textbf{negocjowane} (prowadzący zwracają się o uszczegółowienie lub modyfikację)
      \item \textbf{odrzucone} (prowadzący podaje merytoryczne/formalne powody, dla których podany temat nie może zostać zrealizowany)
      \end{enumerate}
  \end{itemize}
   \begin{alert}{UWAGA!}
     Zgłoszenia znacząco niepełne nie otrzymują statusu negocjowanych (nie ma np. rezerwacji techniki)
   \end{alert}
\end{frame}

\begin{frame}{Wymagania dotyczące raportu}
  \begin{itemize}
  \item Wyraźnie sformułowana hipoteza badawcza lub cel
  \item Krótki wstęp teoretyczny wraz z odniesieniami do istotnej literatury
  \item Literatura: co najmniej 3 artykuły angielskojęzyczne
  \item Opis i źródło użytych danych
  \item Sformułowany model
  \item Wyniki przeprowadzonej empirycznej analizy wraz z~komentarzem, wynikającymi z nich wnioskami oraz odniesieniem do literatury
  \end{itemize}  
  \end{frame}

\begin{frame}{Czego nie proponować}
  \begin{itemize}
  \item Badanie ma być ćwiczeniem z wykorzystaniem ``realnych'' danych, tak jak w życiu zawodowym
  \item Ważne jest jak Państwo sobie radzą i reagują na pojawiające się problemy, a nie spełnienie wszystkich założeń techniki
  \item (Realność danych) Prosimy nie korzystać ze zbiorów danych pochodzących z~podręczników, tutoriali, stron wykładowców
  \item (Antyplagiat) Prosimy nie korzystać z Diagnozy Społecznej (badanie przerwane) ani powtarzać analiz przeprowadzanych podczas zajęć
  \end{itemize}
\end{frame}

\begin{frame}{Wymagania dotyczące raportu -- strona techniczna}
  \begin{itemize}
  \item Raport w formacie pdf
  \item Raport powinien być wyczerpujący -- forma artykułu, a nie logu z programu
  \item Oprogramowanie dowolne (ostrożnie z SaaS) -- ważna jest poprawność uzyskanych wyników i ich opis
  \item Prosimy również zadbać o schludność raportu (brak literówek, poprawność odwołań do literatury, estetykę tabel i opisów...)
  \item 30 000 znaków (ok. 18 stron bez obrazków), przekroczenie limitu należy zgłosić i uzasadnić wykładowcom
  \item Nie czytamy prac anonimowych i bez ponumerowanych stron
  \end{itemize}
\end{frame}

\begin{frame}{Prezentacje}
  \begin{itemize}
  \item Prezentacje mają za zadanie zasugerować Państwu, co należałoby poprawić przed oddaniem końcowego raportu
  \item \textbf{Nie oceniamy zawartości merytorycznej prezentacji}, jedynie technikę prezentacji
  \item Pokazanie wyników badania zapewnia więcej komentarzy -- warto je pokazać \underline{nawet}, jeśli wiedzą Państwo, że są nie do końca prawidłowe lub błędne
  \item Od Państwa zależy podział zadań w grupie (\textbf{nie każda osoba musi prezentować}) -- grupa oceniana jest jako całość
  \item Jeśli zasugerowane merytoryczne poprawki zostaną uwzględnione w końcowym raporcie wcześniejsze błędy/uchybienia nie mają znaczenia
  \item Prezentacji nie poprawiamy w drugim terminie (publiczność!)
  \end{itemize}
\end{frame}

\begin{frame}{Prezentacje -- na co zwracać uwagę}
  \begin{itemize}
  \item Prezentacja != odczyt
  \item Staramy się mówić do sali (rozumiemy, że np. przy wynikach wygodniej jest spojrzeć na rzutnik)
  \item Generalnie \emph{less is more} na slajdach, ale umiar jest jeszcze lepszy
  \item Nie tracimy czasu na to, co publiczność wie -- technikę omawiamy tylko, jeśli istotnie wykracza poza zakres omawiany podczas zajęć
  \item Prezentacja powinna trwać ok. 20 min (z tolerancją 4 min.)
  \end{itemize}
\end{frame}

\begin{frame}{Na co zwrócić uwagę przy opisie badania?}
  \begin{itemize}
  \item Czy temat jest ważny?
  \item Czy temat jest osadzony w teorii ekonomii lub czy jest to badanie interdyscyplinarne (tło teoretyczne)?
  \item Czy hipotezy są weryfikowalne i poprawnie sformułowane?
  \item Opis bazy danych
  \item Szacowanie parametrów modelu
  \item Diagnostyka, określenie ogólności wyników, ograniczenia badania
  \item Weryfikacja hipotez (o ile są)
  \item Interpretacja wyników, wnioski, odniesienie do literatury
  \end{itemize}
\end{frame}

\begin{frame}{Terminarz}
  \begin{enumerate}
  \item Zgłoszenie składu osobowego grup i tematów prac zaliczeniowych: \textbf{6~listopada 20:00 - 26 listopada 20:00}
  \item Przesłanie wybranej bazy danych \underline{do obojga prowadzących}: \textbf{6~listopada 20:00 -  30~listopada 20:00} (jeśli baza jest ogólnodostępna wystarczy link i wskazanie zmiennych)
  \item Prezentacje w kolejności tematów zgodnej z omówieniem ich na zajęciach: \textbf{7, 14, 21 stycznia}
  \item Przesłanie finalnej wersji raportu oraz dostarczenie papierowej wersji pracy na WNE: \textbf{nie później niż w 7 dniu po terminie prezentacji}
  \end{enumerate}
\end{frame}

\section{O czym będą te zajęcia?}

\begin{frame}{Kilka słów o analizie wielowymiarowej}
  \begin{itemize}
  \item Analiza wielowymiarowa to zbiór metod i technik analizy danych zawierających informacje o wielu obiektach opisanych jednocześnie za pomocą wielu zmiennych
  \item Jej celem jest redukcja dużego zbioru danych, uproszczenie ich struktury oraz zapewnienie przejrzystej interpretacji wyników
  \item Analizę wielowymiarową można również zastosować do sortowania, grupowania, skupiania obiektów wykazujących podobne cechy
  \item Dodatkowo za jej pomocą można badać zależności pomiędzy zmiennymi, ich siłę powiązań oraz wyciągać wnioski
  \item Jest to również warsztat narzędzi przydatnych podczas analizy zmiennych jakościowych
  \end{itemize}
\end{frame}

\begin{frame}{Omawiane techniki analizy}
  \begin{itemize}
  \item Powtórzenie statystyki oraz testy (nie)parametryczne
  \item Analiza korelacji
  \item Analiza wariancji i kowariancji (ANOVA)
  \item Analiza kanoniczna
  \item Analiza korespondencji
  \item Analiza dyskryminacji
  \item Metody grupowania i segmentacji
  \item Analiza czynnikowa, inne techniki redukcji wymiaru ($\star$)
  \item Analiza conjoint
  \end{itemize}
\end{frame}

\end{document}
