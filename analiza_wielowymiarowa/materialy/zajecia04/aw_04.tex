\documentclass{beamer}

\mode<presentation> {
  \definecolor{frameheadforeground}{RGB}{169,33,62}
  \definecolor{frameheadbackground}{RGB}{255,255,255}

  \usetheme{Warsaw}
  %\setbeamercovered{transparent}
}

\setbeamercolor{structure}{fg=frameheadforeground,bg=frameheadbackground}

\usepackage[utf8]{inputenc}
\usepackage[MeX]{polski}

\begin{document}

\begin{frame}
\title[Tytuł]{Analiza kanoniczna}

\author{Dorota Celińska-Kopczyńska, Paweł Strawiński}
\institute{Uniwersytet Warszawski}

\titlepage
\end{frame}
\begin{frame}[allowframebreaks]
\frametitle{Plan zajęć}
  \tableofcontents
\end{frame}

\section{Wprowadzenie}

\begin{frame}{Wprowadzenie}
  \begin{itemize}
  \item Analiza kanoniczna jest uogólnieniem liniowej regresji wielorakiej na dwa zbiory zmiennych (np. zbioru zmiennych objaśniających $X_i$ i zbioru zmiennych objaśnianych $Y_i$)
   \item Technika polega na interpretacji zależności pomiędzy dwoma typami nowych zmiennych: zmiennymi kanonicznymi
   \item Pierwszy typ zmiennych kanonicznych jest liniową funkcją pierwszego zbioru zmiennych wejściowych, a drugi liniową funkcją drugiego zbioru zmiennych wejściowych
   \item Zmienne kanoniczne mają maksymalnie wyjaśniać zależności liniowe pomiędzy obydwoma zbiorami zmiennych
  \end{itemize}
\end{frame}

\begin{frame}{Pary zmiennych kanonicznych}
  \begin{itemize}
  \item Cel to maksymalizacja kwadratu współczynnika korelacji między zmiennymi kanonicznymi
  \item Pierwsza para zmiennych kanonicznych wyjaśnia \underline{większość} związków pomiędzy zbiorami zmiennych wejściowych, ale żeby w pełni opisać związki pomiędzy nimi potrzeba wyznaczyć kolejne pary zmiennych
  \item Żadna ze zmiennych należących do kolejnej pary zmiennych kanonicznych nie jest skorelowana z żadną ze zmiennych kanonicznych tego samego typu, gdyż wyjaśnia zależności między zbiorami zmiennych wejściowych w innych wymiarach
  \item Korelacje pomiędzy kolejnymi parami zmiennych kanonicznych są coraz słabsze
  \end{itemize}
\end{frame}

\begin{frame}{Przykłady pytań i problemów badawczych}
  \begin{itemize}
  \item Jak \textbf{wyniki z egzaminów maturalnych} studenta wpływają na jego \textbf{wyniki z przedmiotów}: mikroekonomia 1; algebra liniowa; podstawy prawa; język angielski?
 \item Jaki jest związek pomiędzy \textbf{charakterystykami respondenta} a jego \textbf{zadowoleniem}: z pracy, z życia osobistego, z sytuacji finansowej?
 \item Jaki jest związek pomiędzy \textbf{parametrami ciała} a \textbf{wynikami} poszczególnych \textbf{testów sprawnościowych}?
  \end{itemize}
\end{frame}

\section{Metoda}

\begin{frame}{Założenia}
  \begin{itemize}
  \item Dysponujemy dwoma zbiorami zmiennych: $Y_i= (Y_1, \dots, Y_n)$ i~$X_i= (X_1, \dots, X_m)$
  \item Naszym zadaniem jest znalezienie takiej kombinacji liniowej
zmiennych ze zbioru $Y_i$, która możliwie najsilniej koreluje ze zmiennymi ze zbioru $X_i$
  \item Oznacza to, ze szukamy wektorów współczynników $a_i$ i $b_i$ takich, że korelacja $a_i'X_i$ i $b_i'Y_i$ jest możliwie największa
\end{itemize}
\end{frame}

\begin{frame}{Tworzenie zmiennych kanonicznych}
  \begin{itemize}
  \item \textbf{Zmienne kanoniczne} to kombinacje liniowe zbiorów zmiennych wejściowych:
  $$ U = A'X_i $$
  $$ V = B'Y_i $$
  \begin{scriptsize}
  \item $U = [u_{li}]$ to macierz zmiennych kanonicznych pierwszego typu; $u_{li}$ to wartość $l$-tej zmiennej w $i$-tym obiekcie
  \item $V = [v_{li}]$ to macierz zmiennych kanonicznych drugiego typu; $v_{li}$ to wartość $l$-tej zmiennej w $i$-tym obiekcie
  \item $A'= [a_{jl}]$ to transponowana macierz wag kanonicznych, $a_{lj}$ to waga kanoniczna $j$-tej zmiennej w zbiorze $X_i$ dla $l$-tej zmiennej kanonicznej pierwszego typu
  \item $B'= [b_{jl}]$ to transponowana macierz wag kanonicznych, $b_{lj}$ to waga kanoniczna $j$-tej zmiennej w zbiorze $Y_i$ dla $l$-tej zmiennej kanonicznej drugiego typu
 \end{scriptsize}
  \end{itemize}
\end{frame}

\begin{frame}{Tworzenie zmiennych kanonicznych: założenia}
  \begin{itemize}
    \item Wektory $u_i$ i $u_j$ są nieskorelowane między sobą
    \item Wektory $v_i$ i $v_j$ są nieskorelowane między sobą
    \item Korelacje $corr(u_i; v_i)$ tworzą nierosnący ciąg odpowiadający możliwie największym cząstkowym korelacjom
  \end{itemize}
\end{frame}

\begin{frame}{Obliczanie wag kanonicznych 1}
  \begin{itemize}
  \item \textbf{Wagi kanoniczne} mają maksymalizować korelację pomiędzy kolejnymi parami zmiennych kanonicznych: \textbf{korelację kanoniczną}
  \item Wyznacza się je w oparciu o łączną macierz korelacji zmiennych:
    $$
    R =
    \begin{bmatrix}
      R_{YY} & R_{XY} \\
      R_{YX} & R_{XX} \\
    \end{bmatrix}
    $$
    \begin{scriptsize}
     \item $R_{YY}$ to macierz korelacji zmiennych objaśnianych $Y_i$
     \item $R_{XX}$ to macierz korelacji zmiennych objaśniających $X_i$
     \item $R_{XY}$, $R_{YX}$ to macierze korelacji obu rodzajów zmiennych
    \end{scriptsize}  
  \end{itemize}
\end{frame}

\begin{frame}{Obliczanie wag kanonicznych 2}
  \begin{itemize}
  \item Na początku poszukujemy wag kanonicznych pierwszej pary zmiennych kanonicznych, ponieważ ta para w największym stopniu wyjaśnia zależności pomiędzy zbiorami $X_i$ i $Y_i$; następnie wag kanonicznych kolejnych par zmiennych kanonicznych
  \item Wagi kanoniczne maksymalizują współczynnik korelacji kanonicznej:
    $$
    r_{u_i,v_i} = \frac{(a'_iR_{XY}b_i)}{\sqrt{(a'_iR_{XX}a_i)(b'_iR_{YY}b_i)}}
    $$
  \end{itemize}
\end{frame}

\begin{frame}{Obliczanie wag kanonicznych 3}
  \begin{itemize}
  \item Wagi kanoniczne wyznacza się poprzez rozwiązanie układów równań jednorodnych o postaci:
    $$
    \left\{
    \begin{array}{ccc}
      (R^{-1}_{XX}R'_{XY}R^{-1}_{YY}R_{YX} - \lambda_iI)a_i = 0 \\
      \\
      (R^{-1}_{YY}R'_{YX}R^{-1}_{XX}R_{XY} - \lambda_iI)b_i = 0
    \end{array}
    $$
    
    \begin{scriptsize}
    \item $\lambda_i$ to pierwiastek charakterystyczny (wartość własna) odpowiedniej macierzy
    \end{scriptsize}  
  \end{itemize}
\end{frame}

\begin{frame}{Obliczanie wag kanonicznych 4}
  \begin{itemize}
  \item Liczba niezerowych pierwiastków charakterystycznych równań wyznacznikowych jest równa $s = min(n, m)$
  \item Po malejącym uporządkowaniu wartości własnych znajdujemy wagi kanoniczne dla kolejnych par zmiennych kanonicznych, wstawiając do układu równań kolejne wartości własne
  \end{itemize}
\end{frame}

\begin{frame}{Rozwiązanie}
  \begin{itemize}
  \item  Wagi kanoniczne określają wkład poszczególnych zmiennych wejściowych w tworzenie zmiennych kanonicznych 
  \item Zmienne kanoniczne danego typu nie są ze sobą skorelowane, dlatego suma kwadratów współczynników korelacji
kanonicznej dla wszystkich par zmiennych kanonicznych stanowi miarę stopnia \textbf{wyjaśnienia zmienności poprzez
  związki liniowe zbioru zmiennych objaśnianych przez zbiór zmiennych objaśniających}
$$R^2 = \sum_{l=1}^{s}r^2_{u_i,v_i}$$
  \end{itemize}
\end{frame}

\section{Diagnostyka}

\begin{frame}{Uwagi praktyczne}
  \begin{itemize}
  \item Analizowane zmienne powinny mieć rozkład wielowymiarowy normalny
  \item W zbiorze danych nie występują obserwacje odstające (miara Cooka, wartości dźwigni, etc.)
  \item Zmienne nie są również współliniowe
  \item Próba musi mieć dostatecznie dużą liczebność (nieformalnie: liczba obserwacji powinna być większa od co najmniej 20$\times$liczba zmiennych)
  \end{itemize}
\end{frame}

\begin{frame}{Określanie liczby par zmiennych kanonicznych -- założenia}
  \begin{itemize}
  \item Zakładamy, że co najmniej $k$ pierwszych korelacji kanonicznych jest istotnych i testujemy hipotezę o istotności ostatnich $s - k$ korelacji kanonicznych
  \item Wykorzystujemy statystykę testową $\chi^2$
  \item Weryfikacja istotności par zmiennych kanonicznych odbywa się w sposób iteracyjny
  \item Jeśli wartość statystyki testowej jest mniejsza od wartości krytycznej przy przyjętym poziomie istotności, to przynajmniej jeden współczynnik korelacji kanonicznej o indeksie $k + 1$ jest istotny
  \end{itemize}
\end{frame}

\begin{frame}{Określanie liczby par zmiennych kanonicznych -- cd}
  \begin{itemize}
  \item Wiedząc, że kolejne korelacje kanoniczne są coraz mniejsze, przyjmujemy na początek procesu weryfikacji $k = 0$ -- co najmniej pierwsza z korelacji kanonicznych jest istotna
  \item Po braku podstaw do odrzucenia hipotezy o istotności, zwiększamy kolejno indeks $k$ o jeden i testujemy istotność kolejnych współczynników korelacji kanonicznej
  \item W ostatecznej analizie uwzględniamy wszystkie pary zmiennych kanonicznych, dla których współczynniki korelacji kanonicznej są istotne
  \end{itemize}
\end{frame}

\section{Interpretacja}

\begin{frame}{Interpretacja wyników}
  \begin{itemize}
\item Żeby zinterpretować zmienne kanoniczne przedstawiamy zbiory zmiennych wejściowych jako kombinacje liniowe
  zmiennych kanonicznych:
  $$ Y_i = CV $$
  $$ X_i = DU $$
  \begin{scriptsize}
    \item $C = [c_{jl}]$ to macierz kanonicznych ładunków czynnikowych, $c_{jl}$ jest kanonicznym
ładunkiem czynnikowym znajdującym się przy $j$-tej zmiennej wejściowej i $l$-tej zmiennej kanonicznej pierwszego typu
    \item $D = [d_{il}]$ to macierz kanonicznych ładunków czynnikowych, $d_{il}$ jest kanonicznym ładunkiem czynnikowym znajdującym się przy $i$-tej zmiennej wejściowej i $l$-tej zmiennej kanonicznej drugiego typu
  \end{scriptsize}  
  \end{itemize}
\end{frame}

\begin{frame}{Interpretacja wyników -- cd}
  \begin{itemize}
  \item Kanoniczne ładunki czynnikowe są współczynnikami korelacji liniowej pomiędzy zmiennymi pierwotnymi a zmiennymi kanonicznymi
    $$c_{jl}= r_{y_j,v_l}; j = 1, 2, \dots , n; l = 1, 2, \dots, s;$$
    $$d_{il}= r_{x_i,v_l}; i = q + 1, q + 2, \dots, n + m; l = 1, 2, \dots, s;$$
  \item Im większa wartosc bezwzględna ładunku czynnikowego, tym większy nacisk należy kłaść na daną zmienną przy interpretacji zmiennej kanonicznej
  \item Przy interpretacji zmiennych kanonicznych bierzemy pod uwagę zmienne wejściowe silnie skorelowane
  \end{itemize}
\end{frame}

\begin{frame}{Wariancja wyodrębniona}
  \begin{itemize}
  \item Dzieląc sumy kwadratów współczynników korelacji danej zmiennej kanonicznej przez liczbę zmiennych wejściowych odpowiedniego typu uzyskujemy wartość \textbf{wariancji wyodrębnionej}
  \item Wartość ta określa jaki procent wariancji zmiennych wejściowych wyjaśnia średnio dana zmienna kanoniczna
    $$\bar{R^2_{u_l}} = \frac{1}{n} \sum_{j=1}^{n} c^2_{jl}, l = 1,2,\dots,s;$$
    $$\bar{R^2_{v_l}} = \frac{1}{m} \sum_{j=n+1}^{n+m} d^2_{jl}, l = 1,2,\dots,s;$$
  \end{itemize}
\end{frame}

\begin{frame}{Współczynniki redundancji}
  \begin{itemize}
  \item \textbf{Współczynniki redundancji} są miarą stopnia wyjaśnienia wariancji zmiennych pierwotnych danego typu przez zmienne kanoniczne drugiego typu:
    $$R^2_{v_l,Y_i} = \bar{R^2_{v_l}}\lambda_l, l = 1, 2, \dots, s;$$
    $$R^2_{u_l,X_i} = \bar{R^2_{u_l}}\lambda_l, l = 1, 2, \dots, s;$$
  \item Czyli dowiadujemy się, na ile nadwymiarowy jest jeden zbiór danych wobec drugiego zbioru danych
  \end{itemize}
\end{frame}

\end{document}
