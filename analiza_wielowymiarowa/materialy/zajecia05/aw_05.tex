\documentclass{beamer}

\mode<presentation> {
  \definecolor{frameheadforeground}{RGB}{169,33,62}
  \definecolor{frameheadbackground}{RGB}{255,255,255}

  \usetheme{Warsaw}
  %\setbeamercovered{transparent}
}

\setbeamercolor{structure}{fg=frameheadforeground,bg=frameheadbackground}

\usepackage[utf8]{inputenc}
\usepackage[MeX]{polski}
\usepackage{multirow}

\begin{document}

\begin{frame}
\title[Tytuł]{Analiza korespondencji}

\author{Dorota Celińska-Kopczyńska, Paweł Strawiński}
\institute{Uniwersytet Warszawski}

\titlepage
\end{frame}
\begin{frame}[allowframebreaks]
\frametitle{Plan zajęć}
  \tableofcontents
\end{frame}

\section{Wprowadzenie}
\begin{frame}{Wprowadzenie}
  \begin{itemize}
  \item Analiza korespondencji należy do technik analizy wielowymiarowej zmiennych jakościowych.
  \item Polega na przeprowadzeniu operacji na \textbf{tabeli wielodzielczej}, czyli tabeli przedstawiającej rozkład obserwacji ze względu na kilka cech jednocześnie.
  \item Analiza ta dostarcza informacji na temat struktury powiązań pomiędzy zmiennymi, a jej graficzna prezentacja wyników umożliwia intuicyjne wnioskowanie odnośnie powiązań zachodzących pomiędzy kategoriami badanych zmiennych. 
  \end{itemize}
\end{frame}

\begin{frame}{Przykład tabeli wielodzielczej}
\begin{tabular}{|l|p{2cm}|p{2cm}|p{2cm}|r|}\hline
\multirow{2}{*}{Płeć} & \multicolumn{3}{|c|}{Liczba wypalonych papierosów}& \multirow{2}{*}{Suma}\\ \cline{2-4}
& $<10$ & 10-20 & $>20$ & \\ \hline
Kobieta & 15 & 20 & 5 & 40\\\hline
Mężczyzna & 10 & 30 & 20 & 60\\\hline
Suma & 25 & 50 & 25 & 100\\\hline
\end{tabular}
\end{frame}

\section{Ogólna charakterystyka}
\subsection{Analiza niezależności}
\begin{frame}{Pojęcie niezależności}
  \begin{itemize}
  \item Przed przeprowadzeniem analizy korespondencji należy upewnić się, że ta technika analizy jest właściwa
  \item Między badanymi zmiennymi musi zachodzić zależność: dążymy do odrzucenia hipotezy zerowej o niezależności badanych zmiennych
  \item Dwa zdarzenia są \textbf{niezależne}, jeśli prawdopodobieństwo wystąpienia ich iloczynu jest równe iloczynowi ich prawdopodobieństw brzegowych:
    $$ P(A \cap B) = P(A)P(B) $$
  \end{itemize}
\end{frame}

\begin{frame}{Test niezależności $\chi^2$}
  \begin{itemize}
  \item Porównuje się częstości zaobserwowanych z częstościami oczekiwanymi, przy założeniu prawdziwości hipotezy zerowej
  \item \textbf{$H_0$} -- zmienne są niezależne; \textbf{$H_1$} -- istnieje związek pomiędzy zmiennymi
  \item Częstości oczekiwane:
    \begin{scriptsize}
    $$E_{ij} = \frac{\sum_{j=1}^{k} n_{j} \sum_{i=1}^{w}n_{i}}{\sum_{i=1}^{w}\sum_{j=1}^k n_{ij}} = \frac{suma\ wiersza * suma\ kolumny}{suma\ calkowita} $$
      k -- liczba kolumn; w -- liczba wierszy
    \end{scriptsize}
  \item Statystyka testowa:
   \begin{scriptsize}
    $$\chi^2 = \sum \frac{(O_{ij}-E_{ij})^2}{E_{ij}} = \sum_{i=1}^{w}\sum_{j=1}^{k} \frac{(n_{ij}-E_{ij})^2}{E_{ij}}$$
     $O_{ij}$ -- obserwowana częstość komórki
    \end{scriptsize}
  \end{itemize}
\end{frame}

\begin{frame}{Ocena siły związku}
  \begin{itemize}
  \item Test $\chi^2$ służy do sprawdzenia, czy pomiędzy zmiennymi występuje zależność. Nie odpowiada natomiast na pytanie, jak silne jest to powiązanie.
  \item Wartości statystyki $\chi^2$ nie można stosować do pomiaru siły związku, gdyż jest ona zależna od \textbf{liczebności próby i rośnie wraz z jej wzrostem}.
  \item Najpopularniejszymi miarami siły związku opartymi na statystyce $\chi^2$ są:
    \begin{enumerate}
    \item Współczynnik korelacji $\phi$;
    \item Współczynnik zbieżności V-Cramera;
    \item Współczynnik kontyngencji Pearsona.
    \end{enumerate}
  \end{itemize}
\end{frame}

\begin{frame}{Współczynnik korelacji $\phi$}
    $$\phi = \frac{n_{11}n_{22} - n_{12}n_{21}}{\sqrt{n_{11}n_{22}n_{12}n_{21}}}\ dla\ tabel\ 2x2$$
  $$\phi = \sqrt{\frac{\chi^2}{n}}\ w\ p.\ p. $$
  \begin{itemize}
  \item W przypadku tablicy 2x2 równy jest współczynnikowi V-Cramera; przyjmuje wartości z przedziału (-1;1).
  \item W przypadku większych tablic przyjmuje wartości z przedziału (0;1).
  \item Wpływ wielkości próby jest eliminowany dzięki podzieleniu statystyki $\chi^2$ przez liczeność próby.
  \end{itemize}
\end{frame}

\begin{frame}{Współczynnik zbieżności V-Cramera}
  $$V = \sqrt{\frac{\chi^2}{n*min(k-1;w-1)}}$$
  \begin{itemize}
  \item $V = 0$ zmienne są niezależne (brak korelacji).
  \item $V = 1$ pomiędzy zmiennymi występuje silna funkcyjna zależność.
  \item $0 < V < 1$ przedział możliwych wartości współczynnika V-Cramera dla tablic większych niż 2x2
  \item $-1 < V < 1$ przedział możliwych wartości współczynnika V-Cramera dla tablic 2x2.
  \end{itemize}
\end{frame}

\begin{frame}{Współczynnik kontyngencji Pearsona}
  $$P = \sqrt{\frac{\chi^2}{\chi^2 +n}} $$
  \begin{itemize}
  \item $P = 0$ zmienne są niezależne (brak korelacji).
  \item $0 < P < 1$ przedział możliwych wartości współczynnnika kontyngencji Pearsona.
  \item Im wartość współczynnika bliższa 1, tym silniejszy związek pomiędzy zmiennymi.
  \end{itemize}
\end{frame}

\subsection{Pojęcie analizy korespondencji}
\begin{frame}{Motywacja}
  \begin{itemize}
  \item Miary siły związku stanowią zaledwie punkt wyjścia w analizie zmiennych jakościowych. Nie mówią one nic o strukturze powiązań pomiędzy zmiennymi.
  \item Analiza korespondencji (a szczególnie jej graficzna prezentacja) umożliwia intuicyjne wnioskowanie o powiązaniach zachodzących pomiędzy kategoriami badanych zmiennych.
  \end{itemize}
\end{frame}

\begin{frame}{Obszar zastosowań}
  \begin{itemize}
  \item Prezentacja graficzna zależności między zmiennymi jakościowymi.
  \item Metoda wspomagania i uzupełniania -- nie jest zamiennikiem dla bardziej formalnych narzędzi statystycznych.
   \item Metoda raczej ,,eksploracyjna'', ułatwiająca stawianie hipotez dla dalszych etapów badania, niż ,,konfirmacyjna''.
  \end{itemize}
\end{frame}

\section{Metoda}
\begin{frame}{Etapy analizy korespondencji}
  \begin{enumerate}
  \item Wyznaczenie profili wierszowych i kolumnowych
  \item Wyznaczenie masy wiersza i kolumny
  \item Obliczenie odległości między wierszami (kolumnami) za pomocą metryk $\chi^2$
  \item Wyznaczenie przeciętnych profili wierszowych i kolumnowych
  \item Redukcja wymiaru przestrzeni
  \item Utworzenie i interpretacja wspólnego wykresu profili wierszowych i kolumnowych
  \end{enumerate}
\end{frame}

\subsection{Profile}
\begin{frame}{Podstawowe macierze}
  \begin{itemize}
  \item Najpierw obliczamy \textbf{macierz korespondencji}. Dzielimy liczebności w poszczególnych komórkach tabeli wielodzielczej przez liczebność całkowitą badanej próby.
  \item Następnie wyznaczamy \textbf{macierze profili wierszowych i~kolumnowych}.
  \item Macierz profili wierszowych otrzymujemy dzieląc poszczególne elementy wierszy macierzy korespondencji przez sumę wszystkich elementów tego wiersza.
  \item Macierz profili kolumnowych otrzymujemy dzieląc poszczególne elementy kolumn macierzy korespondencji przez sumę wszystkich elementów tej kolumny.
  \end{itemize}
\end{frame}

\begin{frame}{Przeciętne profile kolumnowe i wierszowe}
  \begin{itemize}
  \item \textbf{Masa wiersza} -- suma elementów danego wiersza macierzy korespondencji.
  \item \textbf{Masa kolumny} -- suma elementów danej kolumny macierzy korespondencji.
  \item \textbf{Przeciętny profil kolumnowy}  -- kolumna o elementach będących masami wierszy.
  \item \textbf{Przeciętny profil wierszowy} -- wiersz o elementach będących masami kolumn.
  \item Przeciętny profil wierszowy można uzyskać, dzieląc podsumowujący wiersz w tabeli wielodzielczej przez ogólną liczebność.
  \end{itemize}
\end{frame}

\begin{frame}{}
  \begin{scriptsize}
    \begin{itemize}
\item Tabela wielodzielcza\\
\begin{tabular}{|c|c|c|c|c|}\hline
\multirow{2}{*}{Płeć} & \multicolumn{3}{|c|}{Liczba wypalonych papierosów}& \multirow{2}{*}{Suma}\\ \cline{2-4}
& $<10$ & 10-20 & $>20$ & \\ \hline
Kobieta & 15 & 20 & 5 & 40\\\hline
Mężczyzna & 10 & 30 & 20 & 60\\\hline
Suma & 25 & 50 & 25 & \textbf{100}\\\hline
\end{tabular}
\item Macierz korespondencji
\begin{tabular}{|c|c|c|c|c|}\hline
\multirow{2}{*}{Płeć} & \multicolumn{3}{|c|}{Liczba wypalonych papierosów}& \multirow{2}{*}{Suma}\\ \cline{2-4}
& $<10$ & 10-20 & $>20$ & \\ \hline
Kobieta & \textbf{0,15} & 0,20 & 0,05 & \textbf{0,40}\\\hline
Mężczyzna & \textbf{0,10} & 0,30 & 0,20 & \textbf{0,60}\\\hline
Suma & 0,25 & 0,50 & 0,25 & 1\\\hline
\end{tabular}
\item Macierz profili wierszowych
\begin{tabular}{|c|c|c|c|c|}\hline
\multirow{2}{*}{Płeć} & \multicolumn{3}{|c|}{Liczba wypalonych papierosów}& \multirow{2}{*}{Suma}\\ \cline{2-4}
& $<10$ & 10-20 & $>20$ & \\ \hline
Kobieta & 0,38 & 0,50 & 0,13 & 0,40\\\hline
Mężczyzna & 0,17 & 0,50 & 0,33 & 0,60\\\hline
Suma & 0,25 & 0,50 & 0,25 & 1\\\hline
\end{tabular}
\end{itemize}
\end{scriptsize}
\end{frame}

\subsection{Odległość $\chi^2$}

\begin{frame}{Definicja odległości $\chi^2$}
  $$\chi^2 = d^2(p,p') = \sum_{j=1}^{w(k)} \frac{(p_j-p_j')^2}{\bar{p_j}} $$
  \begin{itemize}
  \item $d(p,p')$ -- odległość między profilami $p$ i $p'$
  \item $p_j$, $p_j'$ -- elementy profilu $p$ i $p'$ (częstości względne)
  \item $\bar{p_j}$ -- elementy przeciętnego profilu
  \item Odległości te obliczane są zarówno dla profili wierszowych, jak i kolumnowych.
  \item Kategorie z relatywnie większą liczbą elementów wywierają mniejszy wpływ na odległość niż kategorie z mniejszą liczbą obserwacji.
  \end{itemize}
\end{frame}

\begin{frame}{Przykład -- obliczenia dla profili wierszowych}
  \begin{itemize}
  \item Macierz profili wierszowych
\begin{tabular}{|p{2cm}|c|c|c|p{2cm}|}\hline
\multirow{2}{*}{Płeć} & \multicolumn{3}{|c|}{Liczba wypalonych papierosów}& \multirow{2}{*}{P. profil kol.}\\ \cline{2-4}
& $<10$ & 10-20 & $>20$ & \\ \hline
Kobieta & 0,38 & 0,50 & 0,13 & 0,40\\\hline
Mężczyzna & 0,17 & 0,50 & 0,33 & 0,60\\\hline
P. profil wiersz. & 0,25 & 0,50 & 0,25 & 1\\\hline
\end{tabular}
  \end{itemize}
  $$\chi^2 = \frac{(0,38-0,17)^2}{0,25} + \frac{(0,5-0,5)^2}{0,5} + \frac{(0,13-0,33)^2}{0,25}=0,34 $$
\end{frame}

\subsection{Rozkład względem wartości osobliwych}

\begin{frame}{Rozkład względem wartości osobliwych}
  \begin{itemize}
  \item Każdy wiersz macierzy profili o wymiarach $wxk$ może zostać przedstawiony jako punkt w przestrzeni k-wymiarowej, generowanej przez kolumny macierzy.
  \item Każda kolumna macierzy profili może zostać przedstawiona jako punkt w przestrzeni w-wymiarowej, generowanej przez wiersze tej macierzy.
  \item Korzystając z analizy korespondencji dążymy do przedstawienia analizowanego zbioru punktów w przestrzeni maksymalnie trójwymiarowej, przy zachowaniu jak największej informacji o zróżnicowaniu wierszy i kolumm.
  \item W tym celu korzysta się z \textbf{rozkładu względem wartości osobliwych} (\emph{Singular Value Decomposition -- SVD}).
  \end{itemize}
\end{frame}

\begin{frame}{Rozkład względem wartości osobliwych -- cd}
  \begin{itemize}
  \item Metoda SVD polega na przedstawieniu macierzy A rzędu r (o~wymiarze w x k) w postaci iloczynu trzech macierzy:
    $$A_{wxk} = U_{wxr} D_{rxr} V_{rxk}$$
    \begin{scriptsize}
    \item U, V -- macierze ortonormalne ($U'U = I_{rxr}$ i $V'V = I_{kxk}$);
    \item D -- macierz diagonalna utworzona z niezerowych wartości własnych macierzy $A'A$, uporządkowanych nierosnąco $\lambda_1 \geq \lambda_2 \geq \dots \geq \lambda_r > 0$
    \item Kolumny macierzy U to wektory własne macierzy $AA'$
    \item Kolumny macierzy V to wektory własne macierzy $A'A$
    \end{scriptsize}
    \item W praktyce z otrzymanego na podstawie tej metody układu współrzędnych do rzutowania interesują nas 2-3 wektory własne.
  \end{itemize}
\end{frame}

\subsection{Bezwładność (Inercja)}
\begin{frame}{Pojęcie bewzładności (inercji)}
  \begin{itemize}
  \item \textbf{Bezwładność (Inercja)} w analizie korespondencji odpowiada pojęciu wariancji.
  \item Całkowita bezwładność to miara roproszenia profili wokół odpowiednich przeciętnych profili. Całkowita bezwładność wierszy pokazuje, jak bardzo poszczególne profile wierszowe różnią się od przeciętnego profilu wierszowego.
  \end{itemize}
  $$\Lambda^2 = \sum m* d^2$$
  \begin{scriptsize}
    gdzie $m$ -- masa wiersza (kolumny); $d^2$ -- kwadrat odległości między profilem wiersza (kolumny) a odpowiednim przeciętnym profilem. 
  \end{scriptsize}
\end{frame}

\begin{frame}{Inercja całkowita}
  \begin{itemize}
  \item Bewzładność dla wierszy jest równa bezwładności dla kolumn. Dlatego najczęściej podaje się tylko jedną wartość nazywaną \textbf{bezwładnością (inercją) całkowitą}.
  \item $\chi^2 = \Lambda^2n$; powiązanie inercji z wartością testu $\chi^2$ wskazuje, że im mniejsza inercja, tym mniejsza szansa wystąpienia istotnego związku między wierszami i kolumnami tabeli wielodzielczej.
  \item Jeśli $\Lambda^2 = 0$ wtedy różnica między profilami a profilem przeciętnym jest niewielka, co oznacza niewielkie roproszczenie wokół profilu przeciętnego. Analogicznie wysoka wartość $\Lambda^2$ oznacza duże rozproszenie wokół profilu przeciętnego.
  \item Maksymalna wartość bezwładności to $min(k, w)-1$.
  \end{itemize}
\end{frame}

\section{Interpretacja}

\subsection{Ocena reprezentatywności danych w przestrzeni dwuwymiarowej}
\begin{frame}{Związek pomiędzy bezwładnością a wartościami własnymi}
  $$\Lambda^2 = \sum_{i=1}^{min(w,k)-1} \lambda^2_i $$
  \begin{itemize}
  \item Dzięki tej zależności możemy wybrać liczbę wymiarów odtwarzających jak najpełniejszą informację zawartą w wyjściowej tablicy kontyngencji
  \item Jeżeli $\frac{\lambda^2_1+\lambda^2_2}{\Lambda^2}$ przyjmuje wartość przekraczającą 0,75 przestrzeń dwuwymiarową można uznać za dobrą reprezentację początkowych danych.
   \item Najlepsza konfiguracja: dwie pierwsze kolumny macierzy V do reprezentacji kolumn i dwie pierwsze kolumny macierzy U do reprezentacji wierszy macierzy kontyngencji.
   \end{itemize}
\end{frame}

\subsection{Interpretacja}
\begin{frame}{Interpretacja wyników}
  \begin{itemize}
  \item Analizujemy położenie punktów obrazujących kategorie z~wierszy i kolumn tablicy kontyngencji.
  \item Jeśli okazuje się, że dwuwymiarowe rozwiązanie zapewnia zadowalające dopasowanie, to kategorie wierszowe, które są bliskie sobie mają zbliżony rozkład (profil) w poszczególnych kolumnach. Analogicznie interpretujemy katogorie kolumnowe.
  \item Badamy rozmieszczenie punktów względem centrum rzutowania oraz punktów odpowiadających kategoriom różnych zmiennych względem siebie.
  \item Kategorie zmiennych położone na wykresie w niedalekiej odległości od siebie wskazują kombinacje pojawiające się częściej niż jest to oczekiwane przy założeniu niezależności między wierszami i kolumnami.
  \end{itemize}
\end{frame}

\end{document}
