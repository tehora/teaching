\documentclass{beamer}

\mode<presentation> {
  \definecolor{frameheadforeground}{RGB}{169,33,62}
  \definecolor{frameheadbackground}{RGB}{255,255,255}

  \usetheme{Warsaw}
  %\setbeamercovered{transparent}
}

\setbeamercolor{structure}{fg=frameheadforeground,bg=frameheadbackground}

\usepackage[utf8]{inputenc}
\usepackage[MeX]{polski}
\usepackage{multirow}

\begin{document}

\begin{frame}
\title[Tytuł]{Analiza conjoint}

\author{Dorota Celińska-Kopczyńska, Paweł Strawiński}
\institute{Uniwersytet Warszawski}

\titlepage
\end{frame}
\begin{frame}[allowframebreaks]
\frametitle{Plan zajęć}
  \tableofcontents
\end{frame}

\section{Wprowadzenie}
\begin{frame}{Wprowadzenie}
  \begin{itemize}
  \item Analiza conjoint jest metodą badania preferencji wyrażonych
  \item Sposób zadawania pytań opiera się na porównywaniu produktów i wyborze bardziej preferowanych rozwiązań, w miejsce pytań wprost o hierarchię ważności cech produktu
  \item Przygotowane scenariusze pozwalają skwantyfikować, na ile poszczególne cechy przyczyniają się do preferowania produktów określonej kategorii
   \item Może być szacowana zarówno na poziomie indywidualnym (respondent), jak i zagregowanym (cała próba)
 \end{itemize}
\end{frame}

\begin{frame}{Intuicja}
  \begin{itemize}
  \item Mamy do zaoferowania kilka propozycji wyjazdów zorganizowanych:
    \begin{itemize}
  \item Mogą być do różnych \textbf{miejsc}: Oslo, Paryż, Sankt Petersburg, Kair
  \item Mogą być w różnych \textbf{cenach}: 1200 zł, 2000 zł, 1500 zł, 1800 zł
  \item Mogą być w różnych \textbf{miesiącach}: styczeń, kwiecień, lipiec, wrzesień
    \end{itemize}
  \item Problem: Co zaproponować naszym klientom? Które cechy powinny być podkreślone na plakatach reklamowych?
    \end{itemize}
\end{frame}

\begin{frame}{Problemy związane z bezpośrednimi deklaracjami}
  \begin{itemize}
  \item Bezpośrednie deklaracje ważności mogą być niedoskonałe:
    \begin{itemize}
    \item Mają małą moc dyskryminacyjną;
    \item Mogą podlegać uwarunkowaniom społecznym;
    \item Ludziom czasem trudno jest określić, co jest dla nich ważne;
    \item ,,Ważność'' jest ogólną miarą nieprzekładalną na konkretniejsze zachowania.
    \end{itemize}
  \item Conjoint mierzy preferencje w momencie wyboru, co jest przekładalne na zachowania konsumentów
  \end{itemize}
\end{frame}

\section{Założenia metody}

\subsection{Atrybuty i poziomy}
\begin{frame}{Atrybuty i poziomy}
  \begin{itemize}
  \item \textbf{Atrybutami} określa się zmienne objaśniające opisujące dobra lub usługi
  \item \textbf{Poziomami} nazywa się realizacje atrybutów -- poszczególne stany wyodrębnione w ramach tego samego atrybutu
  \item Atrybuty i ich poziomy generują różne warianty dóbr nazywane \textbf{profilami}
  \item Liczba wszystkich możliwych do wygenerowania profilów to iloczyn liczby profilów wszystkich atrybutów
  \end{itemize}
\end{frame}

\begin{frame}{Cechy poziomów}
  \begin{itemize}
  \item Badane poziomy powinny być:
    \begin{itemize}
    \item \emph{jednoznacznie} zdefiniowane
    \item \emph{niezależne} -- zawieranie się jednych atrybutów w innych uniemożliwia (lub znacząco utrudnia) interpretację wyników
    \item \emph{wzajemnie wykluczające się} w ramach jednego atrybutu
    \item obejmować możliwe rozwiązania \emph{w całej rozpiętości}
    \end{itemize}
  \item ograniczenia mogą dotyczyć tylko łączenia poziomów atrybutów, które wzajemnie się wykluczają
  \end{itemize}
\end{frame}

\begin{frame}{Efekt liczby poziomów}
  \begin{itemize}
  \item Za duża liczba poziomów powoduje problemy -- granice możliwości poznawczych człowieka
  \item Im więcej poziomów tym atrybut staje się ważniejszy -- należy starać się, aby liczba poziomów była mniej więcej podobna dla wszystkich atrybutów
  \item Należy ograniczać liczbę poziomów dla atrybutów cenowych. Zakłada się, że nie powinna ta liczba przekraczać 5-6 poziomów
  \end{itemize}
\end{frame}

\subsection{Użyteczności i ważności}

\begin{frame}{Użyteczności}
  \begin{itemize}
  \item Użyteczności mierzone są na skalach interwałowych -- poziomy są preferowane o x punktów użyteczności
  \item Punkt odniesienia jest arbitralny, dlatego nie możemy porównywać wartości liczbowych użyteczności między atrybutami
  \item Przyjmuje się, że suma użyteczności poziomów danego atrybutu wynosi 0
  \item Poziomy, które uzyskały użyteczności mniejsze niż 0 nie są ,,odrzucane'' tylko mniej preferowane
  \item Wnioskowanie na temat rozkładu preferencji na podstawie uśrednionych lub sumowanych użyteczności może prowadzić do błędnych wniosków!
  \end{itemize}
\end{frame}

\begin{frame}{Ważności}
  \begin{itemize}
  \item Ważności mierzone są na skali ilorazowej od 0 do 100
  \item Można powiedzieć, że atrybut o ważności 20\% jest dwa razy ważniejszy niż atrybut o ważności 10\%
  \item Ważności atrybutów oblicza się jako różnice użyteczności najbardziej i najmniej preferowanego poziomu i dzieli przez sumę wszystkich różnic (dla wszystkich poziomów)
  \end{itemize}
\end{frame}

\section{Procedura conjoint}

\begin{frame}{Etapy procedury conjoint}
  \begin{enumerate}
  \item Specyfikacja zadania badawczego -- wybór atrybutów i ich poziomów
  \item Określenie postaci modelu i wykorzystywanej metody conjoint
  \item Gromadzenie danych -- jaka część profili jest ukazywana i jak profile są generowane?
  \item Prezentacja profilów -- opis słowny, rysunek, produkt fizyczny i forma badań -- CAWI, PAPI, TI?
  \item Wybór skali pomiaru preferencji -- skale niemetryczne i metryczne
  \item Estymacja modelu -- modele nieparametryczne, MNK, modele wyborów dyskretnych
  \item Analiza i interpretacja wyników
  \end{enumerate}
\end{frame}

\section{Podział metod conjoint}
\subsection{Full Profile conjoint (FP)}
\begin{frame}{Full Profile conjoint (FP)}
  \begin{itemize}
  \item Metoda oparta na ocenie i szeregowaniu poszczególnych kart opisujących oferty
  \item produkt opisywany jest wszystkimi badanymi cechami -- w każdej ofercie występuje inna kombinacja poziomów wszystkich cech
  \item Kombinacje poziomów każdego z atrybutów nie są skorelowane, co pozwala oszacować wpływ każdego poziomu
  \item Respondent dokonuje oceny chęci zakupu na skali lub szeregowania ofert
  \end{itemize}
\end{frame}

\begin{frame}{Full Profile Conjoint -- ograniczenia}
  \begin{itemize}
  \item Technika nie powinna być stosowana dla większej liczby atrybutów niż 6-7
  \item Minimalna liczba kart to
    $$ min = liczba\ poziomow - liczba\ atrybutow + 1$$
  \item Dla osiągnięcia stabilnych oszacowań dla każdego respondenta zaleca się trzykrotność minimum pokazanych kart
  \end{itemize}
\end{frame}

\subsection{Adaptive conjoint (ACA)}
\begin{frame}{Adaptive conjoint (ACA)}
  \begin{itemize}
  \item Technika korzysta z czterech rodzajów pytań, na podstawie których wyliczane są użyteczności każdego badanego poziomu i ważność cech:
    \begin{enumerate}
    \item \textbf{rangowanie poziomów} -- respondent porządkuje poziomy wewnątrz każdej cechy
    \item \textbf{dystans między najbardziej i najmniej preferowanym} -- jak ważna jest różnica między preferowanymi poziomami
    \item \textbf{porównywanie parami} -- profile złożone z kilku (2-3) cech na raz, skupiające się na cechach najważniejszych dla respondenta
    \item deklarowanie \textbf{chęci zakupu} produktów najbardziej, najmniej i pośrednio preferowanych
    \end{enumerate}
  \end{itemize}
\end{frame}

\begin{frame}{Adaptive conjoint -- zalety}
  \begin{itemize}
  \item Jedyna technika pozwalająca na analizę więcej niż 6 cech lub gdy cechy są wielopoziomowe
  \item Umożliwia zebranie dużej liczby informacji w krótkim czasie
  \item Wywiad ACA jest z reguły przez respondentów postrzegany jako krótszy i bardziej interesujący niż FP
  \item Użyteczności policzone metodą ACA są bardziej wyraziste i rzadziej są źle ukierunkowane niż FP
  \item Użyteczności cząstkowe są bardziej stabilne, mniej wrażliwe na wykluczenie pewnych kombinacji
  \end{itemize}
\end{frame}

\begin{frame}{Adaptive conjoint -- ograniczenia}
  \begin{itemize}
  \item Możemy podwójnie liczyć cechy, które nie są w pełni niezależne
  \item Respondenci mogą nie pamiętać, że wszystkie pozostałe cechy porównywanych produktów są takie same
  \item Metoda silnie bazuje na wstępnych deklaracjach ważności, łamie więc zasady analizy, wedle których ważności te mają być dopiero oszacowane; pojawiają się problemy w przypadku niespójnych deklaracji respondentów
  \item Może niedoszacowywać ważności ceny -- FP i CBC uważane są za lepsze do badań cenowych, ale można ważyć ważność ceny
  \end{itemize}
\end{frame}

\subsection{Choice Based Conjoint (CBC)}
\begin{frame}{Choice Based Conjoint (CBC)}
  \begin{itemize}
  \item Metoda ma symulować realne sytuacje rynkowe, gdy respondent staje przed koniecznością dokonania wyboru z zestawu oferowanych produktów
  \item Najpierw definiuje się zestawy cech, a następnie plan eksperymentu, czyli tabelę opisującą, które produkty będą kolejno zestawiane z którymi
  \item W każdym kroku respondentowi prezentowana jest krótka lista produktów, z których musi wybrać najbardziej preferowany albo wskazać opcję ,,nie kupię niczego'' -- niezainteresowani nie są zmuszani do wyboru
  \item Metoda pozwala mierzyć siłę wpływu poszczególnych cech na dokonywane wybory, jak również siłę wpływu interakcji tych cech
  \end{itemize}
\end{frame}

\begin{frame}{Choice Based Conjoint -- zalety}
  \begin{itemize}
  \item Sytuacje symulują rynkową rzeczywistość, jest też opcja ,,nie wybieram niczego''
  \item Pozwala mierzyć główne efekty, jak i interakcje cech
  \item Metoda wymaga stosunkowo niewielu decyzji podejmowanych przez respondenta (nawet do 20), aby pomiar był rzetelny
  \item Najczęściej wykorzystywana do badań cenowych/promocji
  \end{itemize}
\end{frame}

\begin{frame}{Choice Based Conjoint -- ograniczenia}
  \begin{itemize}
  \item Wybór jest nieefektywnym pomiarem -- wskazuje, co jest preferowane, ale nie wiadomo, na ile
  \item Wymaga zwykle większych prób niż ACA
  \item Zadanie jest bardziej złożone, więc respondent może przetwarzać mniej atrybutów (maksymalnie 6)
  \item Bardziej złożone zadania skłaniają do uproszczonej strategii decyzyjnej (np. tylko cena)
  \end{itemize}
\end{frame}

\subsection{Holdout Choice Scenarios}
\begin{frame}{Holdout Choice Scenarios}
  \begin{itemize}
  \item Oferty wyglądają jak wybory CBC, ale nie są wykorzystywane do wyliczania użyteczności
  \item Sprawdzamy, jaki odsetek respondentów wybrał każdą z opcji
  \item Te same opcje definiowane są następnie w symulatorze w celu sprawdzenia przewidywanych przez model udziałów
  \item Porównanie obserwowanych i przewidywanych udziałów pozwala na ocenę modelu
  \end{itemize}
\end{frame}

\section{Podsumowanie}
\begin{frame}{Mity wokół analizy conjoint}
  \begin{itemize}
  \item Analiza nie udzieli informacji na temat \textbf{udziałów w rynku} badanych produktów -- może pokazać w uproszczony sposób szanse produktu na rynku, ale nie będzie prognozą sprzedaży
  \item Analiza nie powie nam o \textbf{dobrych i złych} cechach produktu, wskaże tylko, jak są względem siebie preferowane wewnątrz zdefiniowanego zbioru
  \item Analiza nie określi \textbf{optymalnej ceny} produktu, udzieli tylko informacji o elastyczności cenowej
  \item Nie zawsze wymagana jest \textbf{próba o dużej liczebności} -- możemy zastosować kilka typów metod conjoint, za to na mniejszej próbie
  \item Conjoint Analysis !$=$ Discrete Choice Experiments!  
  \end{itemize}
\end{frame}

\end{document}
