\documentclass[12pt]{article}

\usepackage[left=2.2cm,top=2.2cm,right=2.2cm,bottom=2.2cm]{geometry}
\usepackage{polski}
\usepackage[utf8]{inputenc}
%\usepackage{indentfirst}
\usepackage{amsmath}
\linespread{1.3}
\makeatletter

\title{instrukcja}
\author{Dorota Celińska}

\begin{document}
\noindent Szanowni Państwo,
\vspace{0.5cm}\\
\noindent zwracam się z prośbą o wypełnienie krótkiej ankiety na potrzeby zajęć z przedmiotu Analiza Wielowymiarowa. Dane zebrane podczas tego badania posłużą jako ilustracja techniki analizy \textit{conjoint} i nie będą publikowane w sposób umożliwiający identyfikację poszczególnych respondentów.

\vspace{0.5cm}
\noindent W niniejszym badaniu zaprezentowane zostanie Państwu 9 możliwych scenariuszy zasad zaliczenia przedmiotu Analiza Wielowymiarowa. Zaliczenie tego przedmiotu może odbywać się na podstawie kilku elementów:
\begin{itemize}
 \setlength{\itemsep}{0cm}%
 \setlength{\parskip}{0cm}%
\item napisania egzaminu końcowego;
\item kartkówek;
\item obligatoryjnych prac domowych;
\item przeprowadzenia badania i sporządzenia z niego raportu.
\end{itemize}

\noindent Każdy z tych elementów może mieć różną wagę podczas oceny końcowej. Oto objaśnienie poszczególnych elementów opisu:

\begin{itemize}
 \setlength{\itemsep}{0cm}%
 \setlength{\parskip}{0cm}%
\item egzamin końcowy może się odbyć (,,tak''), ale można z niego zrezygnować (,,nie'');
\item kartkówek może w ogóle nie być (,,brak''), może być ich mało (,,1-3''), można też przeprowadzać kartkówkę po omówieniu niemalże każdej techniki analizy (,,4-7'');
\item obligatoryjnych prac domowych może w ogóle nie być (,,brak''), może być ich mało (,,1-2''), a można zadawać pracę domową po omówieniu niemalże każdej techniki analizy (,,3-5'');
\item w trakcie semestru nie trzeba przeprowadzać własnego badania (,,brak''), ale można takie przygotować indywidualnie (,,indywidualny'') albo w grupach (,,grupowy'').
\end{itemize}

\noindent Przykładowy scenariusz:
\begin{verbatim}
Egzamin_koncowy;Kartkowki;Prace_domowe;Raport
"tak";"brak";"3-5";"brak"
\end{verbatim}
\noindent oznacza, że zaliczenie odbywa się na podstawie prac domowych zadawanych po omówieniu niemalże każdej techniki analizy i przystąpienia do egzaminu końcowego.

\vspace{0.5cm}\\
\noindent \textbf{Państwa zadaniem jest określenie dla każdego scenariusza, jak bardzo jest on przez Państwa preferowany w skali od 1 (,,najmniej preferowany'') do 10 (,,najbardziej preferowany'').} Każdą ocenę dla każdego scenariusza proszę wpisać w polu ,,Ocena''. Proszę nie zmieniać kolejności scenariuszy!
\newpage

\noindent Przykład:
\begin{verbatim}
Egzamin_koncowy;Kartkowki;Prace_domowe;Raport;Ocena
"tak";"4-7";"3-5";"indywidualny";1
"tak";"brak";"brak";"grupowy";4
"nie";"brak";"1-2";"brak";10
\end{verbatim}

\noindent Respondent uznał scenariusz 1 za najmniej preferowany, dlatego wystawił mu ocenę 1. Drugi scenariusz określony został jako ,,słabo preferowany'' z oceną 4. Ostatni scenariusz był najbardziej preferowany -- ocena 10.

\vspace{0.5cm}

\noindent \textbf{Rozwiązania w postaci uzupełnionego o oceny pliku .csv proszę przesłać do 16 listopada, godziny 20:00 na adres dcelinska@wne.uw.edu.pl. Osoby, które wykonają to zadanie, otrzymają dodatkowe 2 punkty do punktacji z prac domowych. Proszę o rzetelne wykonanie zadania -- zostanie sprawdzone, czy nie występują podejrzane duplikaty.}
\end{document}
